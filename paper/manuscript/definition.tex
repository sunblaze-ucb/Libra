%!TEX root = fastZKP.tex

\section{Zero knowledge proof}\label{ZKP}

We have learned the interactive proof in the preliminary. We would introduce the zero knowledge proof system based on the interactive proof system in this section.\\

\noindent
\textbf{Argument Systems.} Let $R$ be an $NP$ relation. An argument system for $R$ is a protocol between
computationally bounded prover $\mathcal{P}$ and a verifier $\mathcal{V}$ at the end of which $\mathcal{V}$ is convinced in the validity of a statement made by $\mathcal{P}$ of the from “there exists $w$ such that $(x; w) \in R$” for some input $x$. In the sequel we focus on arguments of knowledge which have the stronger property that if the prover manages to convince
the verifier of the statement’s validity, then the prover must know $w$. We use $\mathcal{G}$ to represent private key $pk$ and verificaiton key $vk$ generation phase. Formally, consider the definition below.

\begin{definition}\label{def::zkpd}

Let $R$ be an NP relation and let $\lambda$ be a security parameter. A tuple of algorithm $(\mathcal{G}, \mathcal{P}, \mathcal{V})$ is a zero knowledge argument fro $R$ if the following holds.

\begin{itemize}

\item \textbf{Correctness}. For every $(vk, pk)$ output by $\mathcal{G}(1^\lambda)$ and $(x, w) \in R$ we have
$$\langle \mathcal{P}(pk, w), \mathcal{V}(vk) \rangle(x) = accept$$

\item \textbf{Soundness}. For any probabilistic polynomial time prover $\mathcal{P^*}$ there exists a probabilisitic polynomial extractore $\mathcal{E}$ which runs on the same randomness as $\mathcal{P^*}$ such that for any $x$ it holds that

$$Pr[\langle\mathcal{P^*}(pk), \mathcal{V}(vk) \rangle(x) = accept \wedge (x, w) \notin R | w \leftarrow \mathcal{E}(pk, x)] \leq neg(\lambda)$$

\item \textbf{Zero knowledge}. There exists a probabilistic polynomial simulator $\mathcal{S}$ such that for any probabilistic polynomail time adversary $\mathcal{V^*}$, auxiliary input $z \in \{0, 1\}^{poly(\lambda)}$ the following holds

$$\{View_{V^*}(\mathcal{P}(x, w) \leftrightarrow \mathcal{V^*}(x, z))\} = \{\mathcal{S}^{\mathcal{V^*}}(x, z)\}$$

where $=$ means perfect zero knowledge. 

\end{itemize}

\end{definition}